\section{Elaborazione digitale del segnale con precisione finita}

\cite[cap 11.1]{diniz2010}
% implementazione in un pc
per l'implementazione in un pc si tratta di un problema di
efficienza computazionale dall'algoritmo per un linguaggio di alto livello ( come il C o matlab)
in temini di velocità di esecuzione e utilizzo della memoria, ma le possibilità di controllo sono
in questo caso limitate.

% implementazione in hardware specifico
hardware specifico per applicazioni con requisiti stringenti di velocità di esecuzione (real-time) e 
risorse limitate ( consumo energetico, memoria).
PLD ( programmable logic devices):
 - FPGA  which represent an intermediate integarted stage between discrete hardware and full-custom integrated circuits or digital signal processor DSP.
 -VLSI (Very large scale integration) ASICs (Application-specific integrated circuits)
 

\section{Errori dovuti alla quantizzazione}

emergono a diversi livelli del processo di elaborazione :

- input signal in livelli discreti
- costanti moltiplicative del filtro approssimazione : veid anche \ref[cap. 4.6.5.]{diniz2010} Sensitivity
- risultato della moltiplicazione per troncamento o arrotondamento

e dipendono dall'aritmetica utilizzata \ref{Rappresentazioni numeriche binarie}

{optimal design} to reduce round-off errors
 
 
\section{Rappresentazioni numeriche binarie}

Numeri
+ a virgola mobile: implementaizone compelssa, richiede la quantizzazione della mantissa sia dopo l'addizione che la moltiplicazione 
+ a virgola fissa 

- sign-magnitude
- complemento a 1
- complemento a 2

Aritmetica distribuita: design alternative for digital filter that elimnates the necessity of multiplier elements.


\section{Elementi base}

(adder, multiplier) x (serial, parallel)

serial arithmetic:
i bit sono elaborati sequenzialmente
\cite[cap 11.1]{diniz2010} very efficient in terms of requeired hardware, power consumption, modularity
main drawback is processing speed

- full adder
- serial multiplier

serial multiplier:
- attenzione ai numeri con molti zeri
- risultato è un numero con M+N+1 bit
- troncamento o arrotondamento
si può ottenere precisione doppia raddoppiando il numero di bit, con zeri alla fine dle numero
\cite[pag. 683]{diniz2010}

la moltiplicazione è un'operazione costosa e lenta. La velocità computazionale di molte
applicazioni è dominata da questa aoperaizone, si sviluppano blocchi appositi, moltiplicatori.

\cite[11.4]{Rabaey2005}

